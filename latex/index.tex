\hypertarget{index_intro_sec}{}\section{Introduction}\label{index_intro_sec}
As datasets continue to grow in size and complexity, exploring multi-\/dimensional datasets remain challenging for analysts. A common operation during this exploration is drilldown—understanding the behavior of data subsets by progressively adding filters. While widely used, in the absence of careful attention towards confounding factors, drill-\/downs could lead to inductive fallacies. Specifically, an analyst may end up being “deceived” into thinking that a deviation in trend is attributable to a local change, when in fact it is a more general phenomenon; we term this the drill-\/down fallacy. One way to avoid falling prey to drill-\/down fallacies is to exhaustively explore all potential drill-\/down paths, which quickly becomes infeasible on complex datasets with many attributes. We present Vis\+Pilot, an accelerated visual data exploration tool that guides analysts through the key insights in a dataset, while avoiding drill-\/down fallacies. Our user study results show that Vis\+Pilot helps analysts discover interesting visualizations, understand attribute importance, and predict unseen visualizations better than other multidimensional data analysis baselines.\hypertarget{index_install_sec}{}\section{Installation}\label{index_install_sec}
\hypertarget{index_step1}{}\subsection{To build the project, run\+:}\label{index_step1}

\begin{DoxyCode}{0}
\DoxyCodeLine{bash build.sh}
\end{DoxyCode}
 Under the {\ttfamily /vispilot/} directory.

Install postgres at\+: \href{https://postgresapp.com/}{\texttt{ https\+://postgresapp.\+com/}}

If it doesn\textquotesingle{}t work on the current port(5432), try a another port. 
\begin{DoxyCode}{0}
\DoxyCodeLine{\$psql -d postgres}
\DoxyCodeLine{}
\DoxyCodeLine{CREATE USER summarization WITH CREATEDB CREATEROLE;}
\DoxyCodeLine{ALTER USER summarization WITH PASSWORD \textcolor{stringliteral}{'lattice'};}
\DoxyCodeLine{ALTER USER summarization WITH SUPERUSER;}
\end{DoxyCode}
\hypertarget{index_step2}{}\subsection{To run the project, run\+:}\label{index_step2}

\begin{DoxyCode}{0}
\DoxyCodeLine{bash run.sh}
\end{DoxyCode}
 Under the {\ttfamily /vispilot/} directory. 